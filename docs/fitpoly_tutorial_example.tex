% Options for packages loaded elsewhere
\PassOptionsToPackage{unicode}{hyperref}
\PassOptionsToPackage{hyphens}{url}
%
\documentclass[
]{article}
\usepackage{lmodern}
\usepackage{amssymb,amsmath}
\usepackage{ifxetex,ifluatex}
\ifnum 0\ifxetex 1\fi\ifluatex 1\fi=0 % if pdftex
  \usepackage[T1]{fontenc}
  \usepackage[utf8]{inputenc}
  \usepackage{textcomp} % provide euro and other symbols
\else % if luatex or xetex
  \usepackage{unicode-math}
  \defaultfontfeatures{Scale=MatchLowercase}
  \defaultfontfeatures[\rmfamily]{Ligatures=TeX,Scale=1}
\fi
% Use upquote if available, for straight quotes in verbatim environments
\IfFileExists{upquote.sty}{\usepackage{upquote}}{}
\IfFileExists{microtype.sty}{% use microtype if available
  \usepackage[]{microtype}
  \UseMicrotypeSet[protrusion]{basicmath} % disable protrusion for tt fonts
}{}
\makeatletter
\@ifundefined{KOMAClassName}{% if non-KOMA class
  \IfFileExists{parskip.sty}{%
    \usepackage{parskip}
  }{% else
    \setlength{\parindent}{0pt}
    \setlength{\parskip}{6pt plus 2pt minus 1pt}}
}{% if KOMA class
  \KOMAoptions{parskip=half}}
\makeatother
\usepackage{xcolor}
\IfFileExists{xurl.sty}{\usepackage{xurl}}{} % add URL line breaks if available
\IfFileExists{bookmark.sty}{\usepackage{bookmark}}{\usepackage{hyperref}}
\hypersetup{
  pdftitle={fitpoly\_example\_rose\_plate13},
  pdfauthor={Jeekin Lau},
  hidelinks,
  pdfcreator={LaTeX via pandoc}}
\urlstyle{same} % disable monospaced font for URLs
\usepackage[margin=1in]{geometry}
\usepackage{color}
\usepackage{fancyvrb}
\newcommand{\VerbBar}{|}
\newcommand{\VERB}{\Verb[commandchars=\\\{\}]}
\DefineVerbatimEnvironment{Highlighting}{Verbatim}{commandchars=\\\{\}}
% Add ',fontsize=\small' for more characters per line
\usepackage{framed}
\definecolor{shadecolor}{RGB}{248,248,248}
\newenvironment{Shaded}{\begin{snugshade}}{\end{snugshade}}
\newcommand{\AlertTok}[1]{\textcolor[rgb]{0.94,0.16,0.16}{#1}}
\newcommand{\AnnotationTok}[1]{\textcolor[rgb]{0.56,0.35,0.01}{\textbf{\textit{#1}}}}
\newcommand{\AttributeTok}[1]{\textcolor[rgb]{0.77,0.63,0.00}{#1}}
\newcommand{\BaseNTok}[1]{\textcolor[rgb]{0.00,0.00,0.81}{#1}}
\newcommand{\BuiltInTok}[1]{#1}
\newcommand{\CharTok}[1]{\textcolor[rgb]{0.31,0.60,0.02}{#1}}
\newcommand{\CommentTok}[1]{\textcolor[rgb]{0.56,0.35,0.01}{\textit{#1}}}
\newcommand{\CommentVarTok}[1]{\textcolor[rgb]{0.56,0.35,0.01}{\textbf{\textit{#1}}}}
\newcommand{\ConstantTok}[1]{\textcolor[rgb]{0.00,0.00,0.00}{#1}}
\newcommand{\ControlFlowTok}[1]{\textcolor[rgb]{0.13,0.29,0.53}{\textbf{#1}}}
\newcommand{\DataTypeTok}[1]{\textcolor[rgb]{0.13,0.29,0.53}{#1}}
\newcommand{\DecValTok}[1]{\textcolor[rgb]{0.00,0.00,0.81}{#1}}
\newcommand{\DocumentationTok}[1]{\textcolor[rgb]{0.56,0.35,0.01}{\textbf{\textit{#1}}}}
\newcommand{\ErrorTok}[1]{\textcolor[rgb]{0.64,0.00,0.00}{\textbf{#1}}}
\newcommand{\ExtensionTok}[1]{#1}
\newcommand{\FloatTok}[1]{\textcolor[rgb]{0.00,0.00,0.81}{#1}}
\newcommand{\FunctionTok}[1]{\textcolor[rgb]{0.00,0.00,0.00}{#1}}
\newcommand{\ImportTok}[1]{#1}
\newcommand{\InformationTok}[1]{\textcolor[rgb]{0.56,0.35,0.01}{\textbf{\textit{#1}}}}
\newcommand{\KeywordTok}[1]{\textcolor[rgb]{0.13,0.29,0.53}{\textbf{#1}}}
\newcommand{\NormalTok}[1]{#1}
\newcommand{\OperatorTok}[1]{\textcolor[rgb]{0.81,0.36,0.00}{\textbf{#1}}}
\newcommand{\OtherTok}[1]{\textcolor[rgb]{0.56,0.35,0.01}{#1}}
\newcommand{\PreprocessorTok}[1]{\textcolor[rgb]{0.56,0.35,0.01}{\textit{#1}}}
\newcommand{\RegionMarkerTok}[1]{#1}
\newcommand{\SpecialCharTok}[1]{\textcolor[rgb]{0.00,0.00,0.00}{#1}}
\newcommand{\SpecialStringTok}[1]{\textcolor[rgb]{0.31,0.60,0.02}{#1}}
\newcommand{\StringTok}[1]{\textcolor[rgb]{0.31,0.60,0.02}{#1}}
\newcommand{\VariableTok}[1]{\textcolor[rgb]{0.00,0.00,0.00}{#1}}
\newcommand{\VerbatimStringTok}[1]{\textcolor[rgb]{0.31,0.60,0.02}{#1}}
\newcommand{\WarningTok}[1]{\textcolor[rgb]{0.56,0.35,0.01}{\textbf{\textit{#1}}}}
\usepackage{graphicx,grffile}
\makeatletter
\def\maxwidth{\ifdim\Gin@nat@width>\linewidth\linewidth\else\Gin@nat@width\fi}
\def\maxheight{\ifdim\Gin@nat@height>\textheight\textheight\else\Gin@nat@height\fi}
\makeatother
% Scale images if necessary, so that they will not overflow the page
% margins by default, and it is still possible to overwrite the defaults
% using explicit options in \includegraphics[width, height, ...]{}
\setkeys{Gin}{width=\maxwidth,height=\maxheight,keepaspectratio}
% Set default figure placement to htbp
\makeatletter
\def\fps@figure{htbp}
\makeatother
\setlength{\emergencystretch}{3em} % prevent overfull lines
\providecommand{\tightlist}{%
  \setlength{\itemsep}{0pt}\setlength{\parskip}{0pt}}
\setcounter{secnumdepth}{-\maxdimen} % remove section numbering

\title{fitpoly\_example\_rose\_plate13}
\author{Jeekin Lau}
\date{1/25/2021}

\begin{document}
\maketitle

{
\setcounter{tocdepth}{2}
\tableofcontents
}
\hypertarget{snpolisher-axiom-to-fitpoly-format}{%
\section{SNPolisher: axiom to fitpoly
format}\label{snpolisher-axiom-to-fitpoly-format}}

Make sure you have R package `SNPolisher' installed

\begin{itemize}
\tightlist
\item
  input file is ``AxiomGT1.summary.txt'' from the Axiom analysis suite
\end{itemize}

\begin{Shaded}
\begin{Highlighting}[]
\KeywordTok{library}\NormalTok{(SNPolisher)}

\KeywordTok{fitTetra_Input}\NormalTok{(}\DataTypeTok{summaryFile=}\StringTok{"AxiomGT1.summary.txt"}\NormalTok{,}
               \DataTypeTok{output.file=}\StringTok{"AxiomGT1.summary.fitTetra.txt"}\NormalTok{)}
\end{Highlighting}
\end{Shaded}

\hypertarget{fitpoly-genotype-calling}{%
\section{fitPoly: genotype calling}\label{fitpoly-genotype-calling}}

Depending if you need to call dosages on families or just cultivars you
may need too set the `pop.parents' and `population' functions

Since this example does not contain mapping populations, there is no
added benefit of adding the population nor pop.parents files.

\hypertarget{load-in-data}{%
\subsection{Load in Data}\label{load-in-data}}

fitPoly package used for calling dosage

doParallel package used for using multiple cores. (caution: Windows
utilizes RAM differently than linux and macOS). doParallel works better
for linux and macOS

\begin{itemize}
\tightlist
\item
  Possible solutions for running large datasets on Windows machines and
  on machines with lower amounts of RAM is to divide your data into
  chunks and run them separately. After running all the data separately,
  you can combine the dosage calls.
\end{itemize}

\begin{Shaded}
\begin{Highlighting}[]
\KeywordTok{library}\NormalTok{(fitPoly)}
\KeywordTok{library}\NormalTok{(doParallel)}
\NormalTok{data1<-}\KeywordTok{read.delim}\NormalTok{(}\StringTok{"AxiomGT1.summary.fitTetra.txt"}\NormalTok{,}
                 \DataTypeTok{sep=}\StringTok{"}\CharTok{\textbackslash{}t}\StringTok{"}\NormalTok{,}\DataTypeTok{stringsAsFactors=}\NormalTok{F)}
\end{Highlighting}
\end{Shaded}

\hypertarget{run-savemarkermodels}{%
\subsection{Run `saveMarkerModels'}\label{run-savemarkermodels}}

What is important

\begin{itemize}
\item
  ploidy was set at 4
\item
  pop.parents and population set at null
\item
  depending if you need plots you can change that however 130000+ images
  takes up a lot of space if you are not going to use it
\item
  ncores = do not set at more cores than you have available
\item
  the resulting file will be called ``plate13\_scores.dat''
\end{itemize}

NOTES:

\begin{itemize}
\item
  This step takes a very long time. There is no progress bar this is how
  to estimate how much longer or percentage of progress done.

  \begin{itemize}
  \item
    There are 137786 probes that fitpoly has to run.
  \item
    There will be a line in the code after `saveMarkerModels' begins

\begin{verbatim}
saveMarkerModels: batchsize = somenumber
\end{verbatim}
  \end{itemize}
\end{itemize}

\begin{itemize}
\item
  Divide the 137786 by the batchsize number to see how many batches it
  will take to finish the run.
\item
  Go to your working directory in explorer and you should see which
  batch has just finished.
\end{itemize}

\begin{Shaded}
\begin{Highlighting}[]
\KeywordTok{saveMarkerModels}\NormalTok{(}\DecValTok{4}\NormalTok{, }\DataTypeTok{markers=}\OtherTok{NA}\NormalTok{, }\DataTypeTok{data=}\NormalTok{data1, }\DataTypeTok{diplo=}\OtherTok{NULL}\NormalTok{, }\DataTypeTok{select=}\OtherTok{TRUE}\NormalTok{,}
                 \DataTypeTok{diploselect=}\OtherTok{TRUE}\NormalTok{, }\DataTypeTok{pop.parents=}\OtherTok{NULL}\NormalTok{, }\DataTypeTok{population=}\OtherTok{NULL}\NormalTok{, }\DataTypeTok{parentalPriors=}\OtherTok{NULL}\NormalTok{,}
                 \DataTypeTok{samplePriors=}\OtherTok{NULL}\NormalTok{, }\DataTypeTok{startmeans=}\OtherTok{NULL}\NormalTok{, }\DataTypeTok{maxiter=}\DecValTok{40}\NormalTok{, }\DataTypeTok{maxn.bin=}\DecValTok{200}\NormalTok{, }\DataTypeTok{nbin=}\DecValTok{200}\NormalTok{,}
                 \DataTypeTok{sd.threshold=}\FloatTok{0.1}\NormalTok{, }\DataTypeTok{p.threshold=}\FloatTok{0.99}\NormalTok{, }\DataTypeTok{call.threshold=}\FloatTok{0.6}\NormalTok{, }\DataTypeTok{peak.threshold=}\FloatTok{0.85}\NormalTok{,}
                 \DataTypeTok{try.HW=}\OtherTok{TRUE}\NormalTok{, }\DataTypeTok{dip.filter=}\DecValTok{1}\NormalTok{, }\DataTypeTok{sd.target=}\OtherTok{NA}\NormalTok{,}
                 \DataTypeTok{filePrefix=}\KeywordTok{paste0}\NormalTok{(}\KeywordTok{getwd}\NormalTok{(),}\StringTok{"/plate13"}\NormalTok{), }\DataTypeTok{rdaFiles=}\NormalTok{F, }\DataTypeTok{allModelsFile=}\NormalTok{T,}
                 \DataTypeTok{plot =}\StringTok{"none"}\NormalTok{, }\DataTypeTok{plot.type=}\StringTok{"png"}\NormalTok{, }\DataTypeTok{ncores=}\DecValTok{6}\NormalTok{)}
\end{Highlighting}
\end{Shaded}

\hypertarget{run-custom-script}{%
\section{Run custom script}\label{run-custom-script}}

\textbf{This custom script has been packaged into a function available
as an R-package} \href{https://github.com/jeekinlau/RoseArrayTools}{This
link takes you to github for the R-package RoseArrayTools}
\href{https://jeekinlau.github.io/RoseArrayTools/RoseArrayTools_Vignette.html}{RoseArrayTools
vignette}

Set following

\begin{itemize}
\item
  num\_ind = 96 or however many individuals you have
\item
  need to have ``array\_snps\_flanking\_order.csv'' file in the same
  working directory folder
\item
  will result in two files output ``compared\_calls.csv'' and
  ``compared\_calls\_kind\_counts.csv''

  \begin{itemize}
  \item
    compared\_calls.csv - contains the dosage calls that have been
    compared to keep the calls that are consistent, have only one probe,
    and discards the probes that differ
  \item
    compared\_calls\_kind\_counts.csv gives a matrix which shows which
    markers have

    \begin{itemize}
    \item
      D - different calls (discarded)
    \item
      S - ``same'' calls that have both probes in agreement
    \item
      O - ``one'' single probe (one probe is called, other is NA)
    \end{itemize}
  \end{itemize}
\end{itemize}

\begin{Shaded}
\begin{Highlighting}[]
\KeywordTok{library}\NormalTok{(data.table)}
\NormalTok{calls<-}\StringTok{ }\KeywordTok{as.matrix}\NormalTok{(}\KeywordTok{fread}\NormalTok{(}\StringTok{"plate13_scores.dat"}\NormalTok{, }\DataTypeTok{select =} \KeywordTok{c}\NormalTok{(}\DecValTok{1}\OperatorTok{:}\DecValTok{3}\NormalTok{,}\DecValTok{12}\NormalTok{)))}
\NormalTok{header_calls<-calls[}\DecValTok{1}\OperatorTok{:}\DecValTok{1000}\NormalTok{,}\DecValTok{1}\OperatorTok{:}\KeywordTok{ncol}\NormalTok{(calls)]}

\NormalTok{num_ind<-}\DecValTok{96}
\NormalTok{ind<-calls[}\DecValTok{1}\OperatorTok{:}\NormalTok{num_ind,}\DecValTok{3}\NormalTok{]}
\NormalTok{num_markers<-}\KeywordTok{nrow}\NormalTok{(calls)}\OperatorTok{/}\NormalTok{num_ind}

\NormalTok{markers<-}\KeywordTok{matrix}\NormalTok{(,num_markers,}\DecValTok{1}\NormalTok{)}
\KeywordTok{colnames}\NormalTok{(markers)<-}\StringTok{"Probes_ID"}

\ControlFlowTok{for}\NormalTok{ (i }\ControlFlowTok{in} \DecValTok{1}\OperatorTok{:}\KeywordTok{nrow}\NormalTok{(markers))\{}
\NormalTok{  markers[i,}\DecValTok{1}\NormalTok{]<-calls[i}\OperatorTok{*}\NormalTok{num_ind,}\DecValTok{2}\NormalTok{]}
  \KeywordTok{print}\NormalTok{(i)}
\NormalTok{\}}


\NormalTok{genocalls<-}\KeywordTok{matrix}\NormalTok{(, num_markers, num_ind}\OperatorTok{+}\DecValTok{1}\NormalTok{)}
\KeywordTok{colnames}\NormalTok{(genocalls)<-}\KeywordTok{c}\NormalTok{(}\StringTok{"Probes_ID"}\NormalTok{,ind)}
\NormalTok{genocalls[,}\DecValTok{1}\NormalTok{]<-markers}

\NormalTok{stringcall<-calls[,}\DecValTok{4}\NormalTok{]}

\NormalTok{genocalls<-}\KeywordTok{t}\NormalTok{(genocalls)}
\NormalTok{genocalls[}\DecValTok{2}\OperatorTok{:}\KeywordTok{nrow}\NormalTok{(genocalls),}\DecValTok{1}\OperatorTok{:}\KeywordTok{ncol}\NormalTok{(genocalls)]<-stringcall}
\NormalTok{genocalls<-}\KeywordTok{t}\NormalTok{(genocalls)}


\CommentTok{### order to right order}
\NormalTok{genocall_order<-}\KeywordTok{as.matrix}\NormalTok{(}\KeywordTok{read.csv}\NormalTok{(}\StringTok{"array_snps_flanking_order.csv"}\NormalTok{))}

\NormalTok{marker_col<-}\KeywordTok{matrix}\NormalTok{(,}\KeywordTok{nrow}\NormalTok{(genocalls),}\DecValTok{1}\NormalTok{)}
\NormalTok{genocalls2<-}\KeywordTok{cbind}\NormalTok{(marker_col,genocalls)}

\ControlFlowTok{for}\NormalTok{ (a }\ControlFlowTok{in} \DecValTok{1}\OperatorTok{:}\KeywordTok{nrow}\NormalTok{(genocalls2))\{}
\NormalTok{  probe<-genocalls2[a,}\DecValTok{2}\NormalTok{]}
\NormalTok{  genocalls2[a,}\DecValTok{1}\NormalTok{]<-genocall_order[}\KeywordTok{which}\NormalTok{(genocall_order[,}\DecValTok{1}\NormalTok{]}\OperatorTok{==}\NormalTok{probe),}\DecValTok{2}\NormalTok{]}
  \KeywordTok{print}\NormalTok{(a)}
\NormalTok{\}}

\NormalTok{genocalls3<-genocalls2[}\KeywordTok{order}\NormalTok{(genocalls2[,}\DecValTok{1}\NormalTok{]),]}

\NormalTok{genocalls<-genocalls3}


\CommentTok{#compares the probes}

\NormalTok{compare_probes<-}\KeywordTok{array}\NormalTok{(,}\DataTypeTok{dim=}\KeywordTok{c}\NormalTok{(}\KeywordTok{nrow}\NormalTok{(genocalls)}\OperatorTok{/}\DecValTok{2}\NormalTok{,}\KeywordTok{ncol}\NormalTok{(genocalls),}\DecValTok{2}\NormalTok{))}

\ControlFlowTok{for}\NormalTok{ (j }\ControlFlowTok{in} \DecValTok{1}\OperatorTok{:}\KeywordTok{nrow}\NormalTok{(compare_probes))\{}
  \ControlFlowTok{for}\NormalTok{ (k }\ControlFlowTok{in} \DecValTok{1}\OperatorTok{:}\KeywordTok{ncol}\NormalTok{(compare_probes))\{}
\NormalTok{    compare_probes[j,k,}\DecValTok{1}\NormalTok{]<-genocalls[j}\OperatorTok{*}\DecValTok{2-1}\NormalTok{,k]}
\NormalTok{    compare_probes[j,k,}\DecValTok{2}\NormalTok{]<-genocalls[j}\OperatorTok{*}\DecValTok{2}\NormalTok{,k]}
\NormalTok{    \} }
  \KeywordTok{print}\NormalTok{(j)}
\NormalTok{\}}

\KeywordTok{write.csv}\NormalTok{(compare_probes[,,}\DecValTok{1}\NormalTok{],}\StringTok{"probe1.csv"}\NormalTok{, }\DataTypeTok{col.names=}\NormalTok{T, }\DataTypeTok{row.names =}\NormalTok{ F)}
\KeywordTok{write.csv}\NormalTok{(compare_probes[,,}\DecValTok{2}\NormalTok{],}\StringTok{"probe2.csv"}\NormalTok{, }\DataTypeTok{col.names=}\NormalTok{T, }\DataTypeTok{row.names =}\NormalTok{ F)}


\NormalTok{probe1<-}\KeywordTok{as.matrix}\NormalTok{(}\KeywordTok{read.csv}\NormalTok{(}\StringTok{"probe1.csv"}\NormalTok{, }\DataTypeTok{header=}\NormalTok{T))}
\NormalTok{probe2<-}\KeywordTok{as.matrix}\NormalTok{(}\KeywordTok{read.csv}\NormalTok{(}\StringTok{"probe2.csv"}\NormalTok{, }\DataTypeTok{header=}\NormalTok{T))}

\NormalTok{compare_probes<-}\KeywordTok{array}\NormalTok{(,}\DataTypeTok{dim=}\KeywordTok{c}\NormalTok{(}\KeywordTok{nrow}\NormalTok{(probe1),}\KeywordTok{ncol}\NormalTok{(probe1),}\DecValTok{2}\NormalTok{))}
\NormalTok{compare_probes[,,}\DecValTok{1}\NormalTok{]<-probe1}
\NormalTok{compare_probes[,,}\DecValTok{2}\NormalTok{]<-probe2}

\NormalTok{compared_calls<-}\KeywordTok{matrix}\NormalTok{(,}\KeywordTok{nrow}\NormalTok{(compare_probes),}\KeywordTok{ncol}\NormalTok{(compare_probes))}

\NormalTok{compare_probes[}\KeywordTok{is.na}\NormalTok{(compare_probes)]<-}\DecValTok{9}

\ControlFlowTok{for}\NormalTok{ (l }\ControlFlowTok{in} \DecValTok{1}\OperatorTok{:}\KeywordTok{nrow}\NormalTok{(compared_calls))\{}
  \ControlFlowTok{for}\NormalTok{(m }\ControlFlowTok{in} \DecValTok{1}\OperatorTok{:}\KeywordTok{ncol}\NormalTok{(compared_calls))\{}
\NormalTok{    p1<-compare_probes[l,m,}\DecValTok{1}\NormalTok{]}
\NormalTok{    p2<-compare_probes[l,m,}\DecValTok{2}\NormalTok{]}
    
    \KeywordTok{ifelse}\NormalTok{(p1}\OperatorTok{!=}\DecValTok{9} \OperatorTok{&}\StringTok{ }\NormalTok{p2}\OperatorTok{!=}\DecValTok{9} \OperatorTok{&}\StringTok{ }\NormalTok{p1}\OperatorTok{==}\NormalTok{p2, compared_calls[l,m]<-p1,}
           \KeywordTok{ifelse}\NormalTok{(p1}\OperatorTok{==}\DecValTok{9} \OperatorTok{&}\StringTok{ }\NormalTok{p2}\OperatorTok{==}\DecValTok{9}\NormalTok{, compared_calls[l,m]<-}\OtherTok{NA}\NormalTok{,       }
                  \KeywordTok{ifelse}\NormalTok{(p1}\OperatorTok{==}\DecValTok{9} \OperatorTok{&}\StringTok{ }\NormalTok{p2}\OperatorTok{!=}\DecValTok{9}\NormalTok{,compared_calls[l,m]<-p2,       }
                         \KeywordTok{ifelse}\NormalTok{(p1}\OperatorTok{!=}\DecValTok{9} \OperatorTok{&}\StringTok{ }\NormalTok{p2}\OperatorTok{==}\DecValTok{9}\NormalTok{,compared_calls[l,m]<-p1,              }
                                \KeywordTok{ifelse}\NormalTok{(p1}\OperatorTok{!=}\DecValTok{9} \OperatorTok{&}\StringTok{ }\NormalTok{p2}\OperatorTok{!=}\DecValTok{9} \OperatorTok{&}\StringTok{ }\NormalTok{p1 }\OperatorTok{!=}\StringTok{ }\NormalTok{p2, compared_calls[l,m]<-}\OtherTok{NA}\NormalTok{,                       }
                                       \OtherTok{NA}\NormalTok{)))))   }
\NormalTok{  \} }
  \KeywordTok{print}\NormalTok{(l)\}}


\ControlFlowTok{for}\NormalTok{ (b }\ControlFlowTok{in} \DecValTok{1}\OperatorTok{:}\KeywordTok{nrow}\NormalTok{(compared_calls))\{}
\NormalTok{    compared_calls[b,}\DecValTok{1}\NormalTok{]<-genocalls3[b}\OperatorTok{*}\DecValTok{2-1}\NormalTok{,}\DecValTok{1}\NormalTok{]}
  \KeywordTok{print}\NormalTok{(b)}
\NormalTok{\}}

\KeywordTok{colnames}\NormalTok{(compared_calls)<-}\KeywordTok{colnames}\NormalTok{(genocalls3)}

\KeywordTok{write.csv}\NormalTok{(compared_calls, }\StringTok{"compared_calls.csv"}\NormalTok{, }\DataTypeTok{col.names =}\NormalTok{ T, }\DataTypeTok{row.names =}\NormalTok{ F)}



\NormalTok{compared_calls<-}\KeywordTok{matrix}\NormalTok{(,}\KeywordTok{nrow}\NormalTok{(compare_probes),}\KeywordTok{ncol}\NormalTok{(compare_probes))}

\CommentTok{#S if the two probes are the same}
\CommentTok{#D if the calls are different}
\CommentTok{#O if the calls have one that is the same}




\ControlFlowTok{for}\NormalTok{ (l }\ControlFlowTok{in} \DecValTok{1}\OperatorTok{:}\KeywordTok{nrow}\NormalTok{(compared_calls))\{}
  \ControlFlowTok{for}\NormalTok{(m }\ControlFlowTok{in} \DecValTok{1}\OperatorTok{:}\KeywordTok{ncol}\NormalTok{(compared_calls))\{}
\NormalTok{    p1<-compare_probes[l,m,}\DecValTok{1}\NormalTok{]}
\NormalTok{    p2<-compare_probes[l,m,}\DecValTok{2}\NormalTok{]}
    
    \KeywordTok{ifelse}\NormalTok{(p1}\OperatorTok{!=}\DecValTok{9} \OperatorTok{&}\StringTok{ }\NormalTok{p2}\OperatorTok{!=}\DecValTok{9} \OperatorTok{&}\StringTok{ }\NormalTok{p1}\OperatorTok{==}\NormalTok{p2, compared_calls[l,m]<-}\StringTok{"S"}\NormalTok{,}
           \KeywordTok{ifelse}\NormalTok{(p1}\OperatorTok{==}\DecValTok{9} \OperatorTok{&}\StringTok{ }\NormalTok{p2}\OperatorTok{==}\DecValTok{9}\NormalTok{, compared_calls[l,m]<-}\OtherTok{NA}\NormalTok{,       }
                  \KeywordTok{ifelse}\NormalTok{(p1}\OperatorTok{==}\DecValTok{9} \OperatorTok{&}\StringTok{ }\NormalTok{p2}\OperatorTok{!=}\DecValTok{9}\NormalTok{,compared_calls[l,m]<-}\StringTok{"O"}\NormalTok{,       }
                         \KeywordTok{ifelse}\NormalTok{(p1}\OperatorTok{!=}\DecValTok{9} \OperatorTok{&}\StringTok{ }\NormalTok{p2}\OperatorTok{==}\DecValTok{9}\NormalTok{,compared_calls[l,m]<-}\StringTok{"O"}\NormalTok{,              }
                                \KeywordTok{ifelse}\NormalTok{(p1}\OperatorTok{!=}\DecValTok{9} \OperatorTok{&}\StringTok{ }\NormalTok{p2}\OperatorTok{!=}\DecValTok{9} \OperatorTok{&}\StringTok{ }\NormalTok{p1 }\OperatorTok{!=}\StringTok{ }\NormalTok{p2, compared_calls[l,m]<-}\StringTok{"D"}\NormalTok{,                       }
                                       \OtherTok{NA}\NormalTok{)))))            }
    
\NormalTok{  \} }
  \KeywordTok{print}\NormalTok{(l)\}}



\KeywordTok{write.csv}\NormalTok{(compared_calls, }\StringTok{"compared_calls_kind_counts.csv"}\NormalTok{)}
\end{Highlighting}
\end{Shaded}

\end{document}
